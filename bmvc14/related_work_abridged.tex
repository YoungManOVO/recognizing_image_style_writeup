\paragraph{Related Work.}
There has been growing interest in computer vision in predicting aesthetic and perceptual qualities of images, including beauty \cite{Datta-ECCV-2006,Marchesotti-BMVC-2013,Murray-CVPR-2012}, memorability \cite{Isola-CVPR-2011}, ``interestingness'' \cite{Dhar-CVPR-2011,Gygli-ICCV-2013}, sentiment based on object content \cite{Borth-MM-2013}, and similarity of compositions \cite{gemert2011}.
There has been some attention paid to predicting photographic style \cite{Murray-CVPR-2012}, but limited to a small number of optical techniques such as ``HDR'' and simple compositional qualities like ``Duotones.''
Several previous authors have developed systems to classify classic painting styles, including \cite{keren2002,shamir2010}.
These works consider only a handful of styles (less than ten apiece), with styles that are visually very distinct, e.g., Pollock vs.~Dal\'{\i}.
These datasets comprise less than 60 images per style, for both testing and training.
Mensink \cite{Mensink2014} provides a larger dataset of artworks, but does not consider style classification.
